\documentclass[MScProposal,oneside]{savehproposal}
%برای پایان نامه دکتری از خط زیر به جای خط بالا استفاده کنید.
%\documentclass[PhDProposal,oneside]{savehproposal}
% اگر نسخه دو رو از پروپوزال می خواهید oneside را حذف کنید.
%\documentclass[MScProposal,twoside]{savehproposal}
%\documentclass[PhDProposal,twoside]{savehproposal}

%\usepackage[colorlinks=true]{hyperref}
\usepackage{amssymb}
\usepackage{biblatex}
\usepackage{xepersian}
\usepackage[framemethod=TikZ]{mdframed}
\usepackage[bottom]{footmisc}

%\usepackage{graphicx}
\addbibresource{}
\شروع{نوشتار}
{\azadlogo}
\شروع{موضوع}
تشخیص پیامک انبوه با دیدگاه یادگیری ماشین و ارایه یک قالب پیامک جدید
\پایان{موضوع}

\شروع{entitle}
 \begin{flushleft}
 	\rm\lr{Mass Short Message Service Detector with Machine Learning Approch and Providing New Short Message Format}
 \end{flushleft}
\پایان{entitle}
\شروع{دانشجو}
\شروع{mdframed}[style=MyFrame]
{\small
نام:~احسان \hspace{2cm}
نام~خانوادگی: عالم~محمدقاسملو \hspace{2cm}
شماره~دانشجویی:~۹۳۰۳۵۷۸۲۴ \hspace{0.5cm}
\hfill\break
رشته~تحصیلی:~کامپیوتر \hspace{0.5cm}
\hfill\break
مقطع~تحصیلی:~کارشناسی~ارشد \hspace{5.5cm}
گرایش:~نرم~افزار \hspace{0.5cm}
\hfill\break
دانشکده:~علوم و تحقیقات \hspace{0.5cm}
\hfill\break
دوره:~روزانه \hspace{8.3cm}
تاریخ~و~سال~ورودی:~مهرماه~۱۳۹۳ \hspace{1cm}
\hfill\break
نشانی~پستی~در~تهران: \hspace{6.5cm}
تلفن:~۰۲۱ \hspace{0.5cm}

نشانی~پستی~در~شهرستان:  \hspace{6cm}
تلفن:~۰۸۶} \hspace{1cm
}
\پایان{mdframed}
\پایان{دانشجو}

\شروع{استاد}
\شروع{mdframed}[style=MyFrame]
{\small
نام:~اسدالله \hspace{3cm}
نام~خانوادگی:~وکیلی \hspace{3cm}
تخصص~اصلی:~معماری~کامپیوتر \hspace{0.5cm}
\hfill\break
آخرین~مدرك~تحصیلی~دانشگاهی:~دکترای~تخصصی Ph.D \hspace{0.5cm}
\hfill\break
تخصص~جنبی:~شبکه های کامپیوتری \hspace{6cm}
حوزوی:~ندارد \hspace{0.5cm}
\hfill\break
رتبه~دانشگاهی:~استاد \hspace{8.3cm}
سمت:~استاد \hspace{0.5cm}
\hfill\break
سنوات~تدریس~كارشناسی~ارشد:~۱۲~سال\hspace{6cm}
نحوه~همكاری:~تمام~وقت~\hspace{0.5cm}
\hfill\break
نشانی:~ساوه, خیابان استاد مطهری پلاک ۱۲۳
\hspace{6cm}
تلفن:~086-222123456\hspace{0.5cm}
}
\پایان{mdframed}
\پایان{استاد}

\newpage
\شروع{مشاور}

\شروع{mdframed}[style=MyFrame]
{\small
نام:~مجید\hspace{2cm}
نام~خانوادگی:~عسکرزاده\hspace{2cm}
تخصص~اصلی:~شبکه های کامپیوتری\hspace{2cm}
\hfill\break
رتبه~دانشگاهی~یا~درجه~تحصیلی:~استادیار\hspace{2cm}
شغل:~استاد\hspace{2cm}
محل~خدمت:~دانشگاه آزاد ساوه\hspace{2cm}
\hfill\break
تعداد~پایان~نامه~ها~ورساله~های~راهنمایی~شده~كارشناسی~ارشد:~۱۲\hspace{0.5cm}
\hfill\break
تعداد~پایان~نامه~ها~ورساله~های~در~دست~راهنمایی~كارشناسی~ارشد:~۱۱\hspace{0.5cm}
}
\پایان{mdframed}

\شروع{mdframed}[style=MyFrame]
{\small
نام:~مریم\hspace{2cm}
نام~خانوادگی:~رستگارپور\hspace{2cm}
تخصص~اصلی:~بینایی ماشین\hspace{2cm}
\hfill\break
رتبه~دانشگاهی~یا~درجه~تحصیلی:~استادیار\hspace{2cm}
شغل:~استاد\hspace{2cm}
محل~خدمت:~دانشگاه آزاد ساوه\hspace{2cm}
\hfill\break
تعداد~پایان~نامه~ها~و~رساله~های~راهنمایی~شده~كارشناسی~ارشد:~۹\hspace{0.5cm}
\hfill\break
تعداد~پایان~نامه~ها~و~رساله~های~راهنمایی~شده~كارشناسی~ارشد:~۱۵\hspace{0.5cm}
}
\پایان{mdframed}
\پایان{مشاور}


\شروع{اطلاعات}
\شروع{mdframed}[style=MyFrame]
{\small
الف:~عنوان~پایان~نامه:~سامانه تشخیص پیامک انبوه
\hfill\break
فارسی
\hfill\break
ب:~نوع~كار~تحقیقاتی:~عملی
\hfill\break
پ:~تعداد~واحد~پایان~نامه~:۳
\hfill\break
ت:~پرسش~اصلی~تحقیق~«~مساله~تحقیق~»:
\newline
چگونه می توان پیامک های تبلیغاتی انبوه را محدود کرد؟
\newline
قالبی برای پیامک ارایه کنید تا پیامک های انبوه قابل تشخیص باشند؟
\newline
آیا یادگیری ماشینی می تواند پیامک های انبوه را بدون توجه به محتوا تشخیص دهد؟
\newline
آیا متن کاوی می تواند پیامک های انبوه ناخواسته را تشخیص دهد؟
}
\پایان{mdframed}
\پایان{اطلاعات}
\newpage
\شروع{مسله}
\شروع{mdframed}[style=MyFrame]
{\small
پیامک ها به صورت گسترده ای در زندگی استفاده می شوند و نقش مهمی در کارهای روزانه ما ایفا می کنند. پیامک یک رسانه محبوب است که برای ارایه خدمات بانکداری الکترونیکی, یادآوری پرداخت قبوض, تماس اضطراری و هشدار و غیره به کار می رود. به طور معمول پیامک ها توسط دستگاه تلفن همراه ارسال می گردد و فقط شامل تعداد محدودی مخاطب است اماپروتکل \lr{GSM~03.40} امکان ارسال پیامک انبوه در لایه کاربرد TCP/IP را فراهم آورده که ارسال پیامک همتا به همتا \footnotemark[1](SMPP) نام گرفته است. امروزه از این قابلیت به صورت تجاری استفاده می شود. دارنده تلفن همراه پیامک های تبلیغاتی ناخواسته دریافت می کند که با توجه به متفاوت بودن منبع پیام صرفا اعمال یک لیست سیاه ساده نمی تواند مانع و جامع باشد. از طرفی پیامک های سرویس های بانکداری الکترونیک و اخبار مطلوب نیز از همان پروتکل ارسال انبوه بهره می برند لذا رویکرد مسدود کردن پروتکل نیز عملا شامل محروم کردن خود از سرویس های مفید این پروتکل است. رویکرد دیگر به کارگیری یادگیری ماشین است. با استفاده از دسته بندی می توان پیامک ها ناخواسته را به صورت خودکار تشخیص داد اما به این رویکرد کامل نیست و ممکن است پیامکی که ناخواسته است را حذف نکند یا پیامک مطلوب را به اشتباه نخواسته تشخیص دهد. از طرفی یادگیری ماشین نیازمند محتوا است و بدون بررسی ماشینی محتوا تشخیص پیامک ناخواسته ممکن نیست. متن کاوی نیز همانند یادگیری ماشین کامل نیست هرچند می تواند پیامک های ناخواسته را به درستی تشخیص دهد ولی متاسفانه به علت احتیاج به حافظه زیاد و زمان اجرای طولانی از نظر هزینه و زمان مقرون به صرفه نیست.قصد داریم در این پژوهش ضمن بررسی دقیق ساختار پیامک, با استفاده از روش های یادگیری ماشین مبتنی بر دسته بندی و خوشه بندی پیامک های انبوه ناخواسته از دیدگاه دارنده خدمات پیامک بررسی کنیم و ساختار جدیدی برای پیامک تعریف کنیم که طبق این ساختار چند ویژگی جدید بر اساس محل ارسال و محتوا به ساختار قبلی اضافه می شود. با اضافه شدن این ویژگی ها یک الگوریتم دسته بندی قادر است بعد از دریافت مجموعه آموزشی با خطای کمتری پیامک مطلوب مصرف کننده خود را تشخیص دهد. در این پژوهش پیامک فقط از جنبه داده بررسی می شود و مباحث مخابراتی و سیگنال مورد بررسی قرار نخواهند گرفت.
}
\پایان{mdframed}
{\scriptsize
1. تحقیق بنیادی پژوهشی است كه به كشف ماهیت اشیاء ، پدیده ها وروابط بین متغیر ها ، اصول ، قوانین و ساخت یا آزمایش تئوری ها ونظریه ها می پردازد و به توسعه مرزهای دانش رشته علمی كمك می نماید.
\newline
2. تحقیق نظری : نوعی پژوهش بنیادی است واز روشهای استدلال و تحلیل عقلانی استفاده می كند و بر پایه مطالعات كتابخانه ای انجام می شود.
\newline
3. تحقیق كاربردی: پژوهشی است كه با استفاده از نتایج تحقیقات بنیادی به منظور بهبود و به كمال رساندن رفتارها ، روش ها ، ابزارها، وسایل ، تولیدات ، ساختارها والگو های مورد استفاده جوامع انسانی انجام می شود.
\newline
4. تحقیق عملی: پژوهشی است كه با استفاده از نتایج تحقیقات بنیادی و با هدف رفع مسائل و مشكلات جوامع انسانی انجام می شود.

}
\پایان{مسله}
\LTRfootnotetext[1]{Short Message Peer to Peer}
\newpage
\شروع{سوابق}
\شروع{mdframed}[style=MyFrame]
{\small
در اختراع \textsuperscript{[1]} US20060168032 یک نود پردازش و مسیریابی پیام های سیگنالی پکت های داده را از طریق شبکه ارتباطات ارسال و دریافت می کند. نود مسیر یابی شامل یک ماژول متمایز پیامک است که تعیین می کند پیام پیامک ناخواسته در حال ارسال شدن به یک دریافت کننده یا جمع از دریافت کنندگان بوده پیامک های ناخواسته دور انداخته می شوند و بنابراین به گیرنده مورد نظرشان تحویل داده نمی شوند. به عنوان نتیجه, مشترک تلفن همراه و گره های شبکه از ترافیک پیامک های ناخواسته ایزوله می شود. ماژول متمایز پیامک همچنین شامل یک واسط تامین است که به کاربران نهایی و اپراتورهای شبکه امکان کنترل معیار تمایز پیامک را می دهند.
\newline
در اختراع  \textsuperscript{[2]}US20060168032
در یک شبکه ارتباطات راه دور روشی برای تشخیص پیام های ناخواسته می دهد. محتوای یک پیام مشکوک بررسی می شود تا تعیین کند که خصوصیات وزن دار و جمع خصوصیات وزن دار پیام از آستانه بیشتر می شود یا خیر. اگر این جمع وزن ها از آستانه بیشتر شد, پیام به عنوان یک پیام ناخواسته تلقی می شود و این موضوع به بررسی انسانی برای بهترکردن کیفیت ضرایب وزن دار و خصوصیاتی که در بررسی استفاده شده نیاز دارد.
}
\پایان{mdframed}
\پایان{سوابق}

\شروع{فرض}
\شروع{mdframed}[style=MyFrame]
{\small
ارایه یک قالب داده برای پیامک انبوه
\newline
تشخیص محتوای پیامک با استفاده از یادگیری ماشین
\newline
متن کاوی پیامک های انبوه
}
\پایان{mdframed}
\پایان{فرض}

\شروع{اهداف}
\شروع{mdframed}[style=MyFrame]
{\small
ارایه یک روش ترکیبی با استفاده از یادگیری ماشین و قالب جدید پیامک برای تشخیص پیامک های انبوه ناخواسته به طوری که نیازی به استفاده از لیست سیاه و سفید نیست و این الگوریتم با توجه به مجموعه آموزشی که دارنده در اختیار آن قرار می دهند نسبت به انتخاب پیامک انبوه ناخواسته از میان پیامک های انبوه دریافت شده اقدام می کند.نتیجه این کار تمامی مراکز پیامک قابل استفاده خواهد بود.
}
\پایان{mdframed}
\پایان{اهداف}


\شروع{کاربرد}
\شروع{mdframed}[style=MyFrame]
{\small
۱-شرکت ارتباطات سیار می تواند الگوریتم یادگیر این پژوهش را در سامانه مرکز پیامک پیاده سازی کند و با در نظر گرفتن خصوصیات دریافت کننده ها به عنوان یک مجموعه آموزشی پیامک های مطلوب آنان را انتخاب کند
\newline
۲-شرکت ارتباطات زیرساخت می تواند با به کار گیری ساختار پیامک در این پژوهش امکان استفاده از الگوریتم یادگیر را فراهم آورد.
\newline
}
\پایان{mdframed}
\پایان{کاربرد}
\newpage
\شروع{جنبه}
\شروع{mdframed}[style=MyFrame]

\پایان{mdframed}
\پایان{جنبه}

\شروع{روش}
\شروع{mdframed}[style=MyFrame]
{\small
برای انجام این پژوهش ابتدا یک مجموعه داده به عنوان پیامک انبوه آموزشی و یک مجموعه داده به عنوان پیامک انبوه بررسی تهیه می شود هردو مجموعه شامل انواع ترکیبات هستند ولی از نظر محتوا مشابه نیستند.ابتدا مجموعه آموزشی را در قالب جدید به الگوریتم یادگیر ارایه می دهیم. بعد از استخراج ویژگی های مجموعه آموزشی با استفاده از الگو مجموعه بررسی را به الگوریتم می دهیم در صورتی که قادر به تشخیص باشد صحت روش ثابت می شود. این روش از نظر دقت مزیتی بر روش های معرفی شده کنونی ندارد و فقط دیدگاه ترکیبی مطرح شده است.
اطلاعات جمع آوری شده از \lr{Machine Learning Repository} شرکت گوگل است. تجزیه و تحلیل اطلاعات نیز با نرم افزار MATLAB صورت می گیرد.
}
\پایان{mdframed}
\پایان{روش}

\شروع{زمانبندی}
\begin{tabular}{|c|c|c|}
	\hline
تاریخ تصویب & از تاریخ & تا تاریخ 
	\\
	\hline
مطالعات کتابخانه ای & --- & ---
	\\
	\hline
جمع آوری اطلاعات & --- & ---
	\\
	\hline
تجزیه و تحلیل داده & --- & ---
	\\
	\hline
نتیجه گیری و نگارش پایان نامه & --- & ---
	\\
	\hline
تاریخ دفاع نهایی & --- & ---
	\\
	\hline
\end{tabular}
\پایان{زمانبندی}

\شروع{منابع}

\small\lr{
[1] Unwanted message (spam) detection based on message content,
Cai, Y. and Qutub,
S. and Sharma,A.,
http://www.google.com/patents/US20060168032,
2006,jul ~27,
Google Patents,US Patent App. 11/018,270
}
\small\lr{
[2] Methods and systems for preventing delivery of unwanted short message service (SMS) messages,
Allison, R.L. and Marsico,
P.J., http://google.com/patents/US6819932,
2004, nov ~16,
Google Patents, US Patent 6,819,932
}
\پایان{منابع}



\پایان{نوشتار}
